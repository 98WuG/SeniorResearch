\documentclass[11pt,letterpaper]{article}

\usepackage{amsmath}
\usepackage{mathtools}
\usepackage{enumitem}
\usepackage{tikz}

\begin{document}
	\setlength{\parskip}{1em}
	\title{Approximate Solution to the Packing Problem with Respect to Knolling Applications}
	\author{Gerald Wu}
	\maketitle
	\tableofcontents
	\pagenumbering{arabic}

	\section{Background}
	\subsection{Context}
	\subsubsection{History of Knolling}
	Knolling began in 1987, with a janitor of Frank Gehry's shop. Frank Gehry designed furniture for a company called Knoll. The janitor, named Andrew Kromelov, started arranging his tools at 90 degree angles and taking pictures of them because he found it aesthetically pleasing. As a tribute to the company he was working for, he named this process knolling.
	\subsection{Current Solutions}
	\subsection{Problems with Current Solutions}
	\subsection{Goals of Research}

	\section{Methodology}
	\subsection{Approach}
	\subsection{Data Sets}
	\subsection{Software}
	\subsection{Libraries}
	\subsection{Hardware}

	\section{Results}
	\subsection{What Happened}
	\subsection{What Worked}
	\subsection{What Didn't Work}
	\subsection{Notable Observations}

	\section{Conclusion}
	\subsection{What Does This Mean?}
	\subsection{Status of the Problem}
	\subsection{Unknowns}
	\subsection{Improvements}

	\section{References}
\end{document}
